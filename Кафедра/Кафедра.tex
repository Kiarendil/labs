\documentclass[12pt]{kiarticle}
\graphicspath{{pictures/}}
\DeclareGraphicsExtensions{.pdf,.png,.jpg,.eps}
%%%
\pagestyle{fancy}
\fancyhf{}
%\renewcommand{\headrulewidth}{ 0.1mm }
\renewcommand{\footrulewidth}{ .0em }
\fancyfoot[C]{\texttt{\textemdash~\thepage~\textemdash}}
\fancyhead[L]{Изучение характеристик $ B $-мезона \hfil}
\fancyhead[R]{\hfil Проектная деятельность, 3 семестр }
\usepackage{multirow} % Слияние строк в таблице
\newcommand
{\un}[1]
{\ensuremath{\text{#1}}}
\newcommand{\eds}{\ensuremath{ \mathscr{E}}}
\usepackage{tikz}
%%% Работа с таблицами
\usepackage{array,tabularx,tabulary,booktabs} % Дополнительная работа с таблицами
\usepackage{longtable}  % Длинные таблицы
\usepackage{multirow} % Слияние строк в таблице

\begin{document}
	
	\begin{titlepage}
		\begin{center}
			\large 	Московский физико-технический институт \\
			(государственный университет) \\
			Факультет общей и прикладной физики \\
			Образовательная программа \\
			"<Фундаментальные взаимодействия и физика элементарных частиц">
			\vspace{0.2cm}
			
			\vspace{4.5cm}
			\large Проектная деятельность в III семестре на тему: \\ \vspace{0.2cm}
			\LARGE \textbf{Изучение характеристик $ B_0 $-мезона}
		\end{center}
		\vspace{2.3cm} \large
		
		\begin{center}
			Работу выполнили: \\
			Букин Кирилл,
			625 группа \\
			Иванов Кирилл,
			625 группа \\
			Семенов Олег,
			625 группа
			\vspace{10mm}		
			
		\end{center}
		
		\begin{center} \vspace{60mm}
			г. Долгопрудный \\
			2017 год
		\end{center}
	\end{titlepage}
	
\section{Введение}

	Мы изучаем спектр поперечных импульсов и распределение по псевдобыстроте  $ B_0 $-мезонов по данным, полученным на детекторе CMS в 2012 году, на энергии 8 ТэВ. А именно, мы обрабатываем  данные, полученные в ходе реакции
	
	\begin{equation}\label{}
	B_0 \st \Psi(2S) \; K_S; \quad K_S \st \pi^+ \; \pi^-; \quad \Psi(2S) \st J/\Psi \; \pi^+ \; \pi^-; \quad J/\Psi  \st \mu^+ \; \mu^- 
	\end{equation}
	
%	\begin{tikzpicture} [scale = 1.8, yshift=2pt]
%		\draw (0,0) node {$ B_0$};
%		\draw [->] (0.2, 0) -- (1.2, 0);
%		\draw (1.3,0) node[anchor=west] {$ \Psi(2S) + K_S$};
%		\draw [->] (1.7, 0.2) -- (2.2, 0.8);
%		\draw (2.3,0.8) node[anchor=west] {$J/\Psi + \pi^+ + \pi^-$};
%		\draw [->] (2.45, 0.2) -- (2.45, -.2);
%		\draw (2.9, - 0.5) node[anchor=west] {$\pi^+ + \pi^-$};
%		\draw [->] (2.5, 0.7) -- (3.2, 0.2);
%		\draw (3.3, 0.1) node[anchor=west] {$\mu^+ + \mu^-$};
%	
%	\end{tikzpicture}
	
	
	
\section{Мотивация}

	Точное измерение параметров прелестных частиц имеет в настоящий момент большое значение, так как $ B $-физика сейчас является важным направлением физики элементарных частиц, например как одно из направлний для поиска отклонений от стандартной модели. $ b $-кварк достаточно тяжелый и достаточно долгоживущий, в отличии от $ t $-кварка, для вероятности существования интересных распадов и возможности их померить.
	В связи с тем, что распад $ b $-кварка, как самого лёгкого кварка третьего поколения, может происходить только с изменением поколения, распады $ b $-кварка легко идентифицируемы, его большая масса также облегчает экспериментальную идентификацию.
	Процессы, происходящие с участием $ b $-кварка, носят жаргонное название «$ B $-физика». Поскольку распады $ b $-кварка в кварки более лёгкого аромата сильно подавлены и происходят в основном через слабые взаимодействия, то в системах, содержащих эту частицу, наблюдаются многие интересные процессы — в частности, нарушение $ CP $-чётности, осцилляции нейтральных $ B $-мезонов, процессы с вкладом петлевых («пингвинных» (англ. penguin diagram) и «ящичных» (англ. box diagram) диаграмм. $ B $-физика может привести к поиску различных отклонений от Стандартной Модели и пути к "<Новой физике">.
	
	Конкретно в нашей работе мы исследуем спектр поперечных импульсов $ b $-мезонов, а также их распределение по псевдобыстротам, как некую вспомогательную работу, чтобы понимать, какими характеристиками вообще обладают $ B $-мезоны, чтобы впоследствии полученные результаты можно было применять в более серьёзных работах.  
	
\section{Устройство детектора}

	Название CMS расшифровывается как Compact Muon Solenoid (Компактный мюонный соленоид). Длина детектора составляет 20 м, а диаметр — 15 м.
	
	Внутри детектора установлено несколько слоев трековых детекторов, затем идут электромагнитный и адронный калориметры, снаружи расположен магнитный соленоид, а затем — массивное ярмо магнита с мюонными камерами.
	
	Ключевым элементом детектора CMS является тяжелый сверхпроводящий магнит. По своей конструкции он напоминает привычный электромагнит с сердечником, только «вывернутый наизнанку». Вместо внутреннего железного сердечника у него есть внешнее железное ярмо, которое не дает линиям магнитного поля расходиться в пространстве, а как бы удерживает их внутри металла. Благодаря такой конструкции единый электромагнит создает сильное магнитное поле как внутри, так и снаружи цилиндра. Внутри цилиндра помещаются трековые детекторы и калориметры, а наружное поле используется для отклонения мюонов. В результате, когда мюон вылетает из центра детектора и пролетает через центральную область и область возвратного поля, он отклоняется сначала в одну сторону, а потом в другую, вычерчивая характерный профиль, похожий на букву «S». Этот профиль, причем для мюонов разных энергий, присутствует на эмблеме CMS.
	
	\subsection{Трековые детекторы}
	Трековые детекторы в детекторе CMS следуют классической схеме. Ближе всего к вакуумной трубе расположен пиксельный детектор. Три цилиндрических слоя имеют радиусы 4, 7 и 11 см и содержат все вместе 65 миллионов отдельных пикселов, каждый размером 100 на 150 микрон.
	
	На больших расстояниях от оси пучка, вплоть до радиуса 130 см, расположены десять слоев кремниевого полоскового детектора. Он содержит свыше 15 тысяч отдельных модулей разного дизайна, насчитывающих вместе 10 миллионов чувствительных полосок, информация с которых считывается 80 тысячами каналов сбора данных. Для оптимизации работы полосковый детектор поддерживается при температуре –20C.
	
	\subsection{Калориметры}
	
	В соответствие со стандартной практикой, в детекторе CMS установлено два типа калориметров: внутренний (электромагнитный) — для измерения энергий электронов и фотонов, и внешний (адронный) — для измерения энергий адронов.
	
	Электромагнитный калориметр CMS сделан на основе тяжелых сцинтилляционных кристаллов вольфрамата свинца, плотность которых больше, чем у стали. Преимущество этого материала по сравнению с другими сцинтилляторами состоит в том, что электроны и фотоны порождают в нём очень короткие ливни с хорошо известными свойствами. Это значит, что измерение энергий частиц будет происходить с высокой точностью и на малых расстояниях, что очень важно для компактного детектора CMS. Слабая сторона этого сцинтиллятора — высокая чувствительность к температуре, поэтому все сто тонн калориметра приходится держать при постоянной температуре, с отклонениями не более десятой доли градуса. Учитывая, что соседние детекторные компоненты (трековый детектор, соленоид и т. д.) требуют для работы свои специальные температурные режимы, задача охлаждения отдельных компонентов тоже становится нетривиальной.
	
	Адронный калориметр должен породить и поглотить адронные ливни, которые по своей природе более протяженные, чем электромагнитные. Поэтому вместить адронный калориметр внутрь относительно компактного соленоида оказалось непростой задачей. На самом деле, несколько слоев адронного калориметра пришлось даже разместить снаружи соленоида для того, чтобы убедиться, что адронный ливень полностью поглотился веществом и нет утечки ливня наружу.
	Мюонные камеры расположены снаружи соленоида, причем они чередуются со слоями железного ярма, по которому «возвращается» магнитное поле. На детекторе CMS используются мюонные детекторы трех типов: дрейфовые трубки, катодные полосковые камеры и камеры с резистивными пластинками. Часть этих камер предназначена для определения координат и времен пролетевших мюонов, а другая часть используется для быстрого мюонного триггера, который должен в режиме реального времени решить, представляет ли это событие что-то интересное с точки зрения мюонов.
	
\section{Реконструкция}
Мы восстанавливаем наш канал из пионов и мюонов. До детекторов долетают: пионы, каоны, протоны, электроны и мюоны. Их импульс мы получаем из данных трекового детектора. Мюоны мы отличаем от остальных частиц тем, что они фиксируются мюонным детектором. Электроны мы отсекаем по данным электромагнитного калориметра. Остальным частицам приписываем массу пиона и считаем, что это был пион.

Мы поступаем именно так, потому что в колориметре наши частицы держатся довольно компактно, и обычно бывает весьма тяжело точно различить конкретные частицы.

\section{Обработка данных}

\subsection{Критерии отбора}
	В ходе обработки данных мы накладываем на них следующие критерии отбора: 
	
	\begin{enumerate}
		\item Масса $ K_S $ мезона отличается от табличной $ m_{K_S} = 497,614 $ МэВ не более чем на 16 МэВ
		
		\item Инвариантная масса: $ |m_{\psi(2S)} - m_{J/\psi \pi^+\pi^-} | < 15 $ Мэв
		
		\item  Угол между линией, соединяющей вершины $ K_S $ и $ B_0 $, и импульс каона: $ |K_SBcos2 — 1| < 0,00005. $
		
		\item Угол между направлением на вершину$  В $-мезона из первичной вершины и его импульсом $ |B_{pv}cos2C_{jp} — 1| < 0,005 $
		
		\item Значимость отлета $ B $-мезона: $ B_{pvdistsignif}2C_{jp} > 7  $
		
		\item Угол между импульсом каона и направлением из вершины распада каона и $ B$-мезона
	\end{enumerate}

\subsection{Фитирование}

После этого для получения более точного числа сигнальных событий мы фитируем распределение по массам $ B $-мезона с помощью распределения Гаусса, а фон фитируем многочленом Берштейна 3-ей степени. 

\subsection{Разбиение на бины}

Теперь разобьем на бины наши распределения по поперечным импульсам и псевдобыстротам и для каждого бина проведем аналогичные фиттирование по массам $ B $-мезона. По полученным числам сигнальных событий в каждом бине построим гистограммы итогового распределения искомых характеристик.

\section{Результат}
	
Таким образом, мы получили распределение по поперечным импульсам и псевдобыстротам $ B $-мезона при данных каналах распада и энергии $ 8 $ ТэВ. Эти две важные характеристики позволяют нам понимать, какое "<поведение"> $ B $-мезона нам вообще стоит ожидать в других реакциях, какие критерии отбора можно на них накладывать, какие именно эксперименты проводить.
	
	
	
\end{document}