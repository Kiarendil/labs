\documentclass[12pt]{kiarticle}
\graphicspath{{pictures/}}
\DeclareGraphicsExtensions{.pdf,.png,.jpg,.eps}
%%%
\pagestyle{fancy}
\fancyhf{}
%\renewcommand{\headrulewidth}{ 0.1mm }
\renewcommand{\footrulewidth}{ .0em }
\fancyfoot[C]{\texttt{\textemdash~\thepage~\textemdash}}
\fancyhead[L]{Лабораторная работа № 6.1 \hfil}
\fancyhead[R]{\hfil Иванов Кирилл, 625 группа }
\usepackage{multirow} % Слияние строк в таблице
\newcommand
{\un}[1]
{\ensuremath{\text{#1}}}
\newcommand{\eds}{\ensuremath{ \mathscr{E}}}
\usepackage{tikz}
%%% Работа с таблицами
\usepackage{array,tabularx,tabulary,booktabs} % Дополнительная работа с таблицами
\usepackage{longtable}  % Длинные таблицы
\usepackage{multirow} % Слияние строк в таблице

\begin{document}
	
	\begin{titlepage}
		\begin{center}
			\large 	Московский физико-технический институт \\
			(государственный университет) \\
			Факультет общей и прикладной физики \\
			\vspace{0.2cm}
			
			\vspace{4.5cm}
			Лабораторная работа № 5.1 \\ \vspace{0.2cm}
			\large (Общая физика: квантовая физика) \\ \vspace{0.2cm}
			\LARGE \textbf{Измерение коэффициента ослабления потока $ \gamma $-лучей в веществе и определение их энергии}
		\end{center}
		\vspace{2.3cm} \large
		
		\begin{center}
			Работу выполнил: \\
			Иванов Кирилл,
			625 группа
			\vspace{10mm}		
			
		\end{center}
		
		\begin{center} \vspace{60mm}
			г. Долгопрудный \\
			2018 год
		\end{center}
	\end{titlepage}
	
	\paragraph*{Цель работы:} С помощью сцинтилляционного счетчика измерить линейные коэффициенты ослабления потока $ \gamma $-лучей в свинце, железе и алюминии; по их величине определить энергию $ \gamma $-квантов.
	
	\section{Теоретическое введение}
	
	Гамма-лучи возникают при переходе возбужденных ядер из одного энергетического состояния в другое, более низкое. Энергия $ \gamma $-квантов обычно заключена между несколькими десятками килоэлектронвольт и несколькими миллионами электрон-вольт. Гамма-кванты не несут электрического заряда, их масса равна нулю. Проходя через вещество, пучок $ \gamma $-квантов постепенно ослабляется. Ослабление происходит по экспоненциальному закону, который может быть записан в двух эквивалентных нормах:
	
	
	
	
\end{document}