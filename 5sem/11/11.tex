\documentclass[12pt]{kiarticle}
\graphicspath{{pictures/}}
\DeclareGraphicsExtensions{.pdf,.png,.jpg,.eps}
%%%
\pagestyle{fancy}
\fancyhf{}
%\renewcommand{\headrulewidth}{ 0.1mm }
\renewcommand{\footrulewidth}{ .0em }
\fancyfoot[C]{\texttt{\textemdash~\thepage~\textemdash}}
\fancyhead[L]{Лабораторная работа №1.1 \hfil}
\fancyhead[R]{\hfil Иванов Кирилл, 625 группа }
\usepackage{multirow} % Слияние строк в таблице
\newcommand
{\un}[1]
{\ensuremath{\text{#1}}}
\newcommand{\eds}{\ensuremath{ \mathscr{E}}}
\newcommand{\ga}{\ensuremath{\gamma}}
\usepackage{tikz}
%%% Работа с таблицами
\usepackage{array,tabularx,tabulary,booktabs} % Дополнительная работа с таблицами
\usepackage{longtable}  % Длинные таблицы
\usepackage{multirow} % Слияние строк в таблице

\begin{document}
	
	\begin{titlepage}
		\begin{center}
			\large 	Московский физико-технический институт \\
			(государственный университет) \\
			Факультет общей и прикладной физики \\
			\vspace{0.2cm}
			
			\vspace{4.5cm}
			Лабораторная работа № 1.1  \\ \vspace{0.2cm}
			\large (Общая физика: квантовая физика) \\ \vspace{0.2cm}
			\LARGE \textbf{ Экспериментальная проверка уравнения Эйнштейна
				для фотоэффекта и определение постоянной Планка }
		\end{center}
		\vspace{2.3cm} \large
		
		\begin{center}
			Работу выполнил: \\
			Иванов Кирилл,
			625 группа
			\vspace{10mm}		
			
		\end{center}
		
		\begin{center} \vspace{60mm}
			г. Долгопрудный \\
			2018 год
		\end{center}
	\end{titlepage}


	\paragraph*{Цель работы:} 
	Исследовать зависимость фототока от величины задерживающего потенциала и частоты падающего излучения, что позволяет вычислить величину постоянной Планка.
	
	\section{Теоретическое введение}
	
	Фотоэффект --- явление испускания электронов фотокатодом, облучаемым светом,  Это явление хорошо объясняется фотонной теорией света. Взаимодействие монохроматического света с веществом можно описывать
	как взаимодействие с веществом частиц, называемых фотонами, которые обладают энергией $ \hbar \omega $ и импульсом $ \hbar\omega/c $. При столкновении фотона с электроном фотокатода энергия отона полностью передается электрону, и фотон прекращает свое существование. Энергетический баланс этого взаимодействия для вылетающих электронов
	описывается уравнением
	
	\begin{equation}\label{energy balance}
	\hbar \omega = E_{max} + W
	\end{equation}
	
	\begin{wrapfigure}{l}{0.3\linewidth}
		\includegraphics[width=\linewidth]{I(V)}
		\caption{Зависимость фототока от напряжения на аноде фотоэлемента}
		\label{ris I(V)}
	\end{wrapfigure}
	
	Здесь $ E_{max} $ ---  максимальная кинетическая энергия электрона после выхода из фотокатода, $ W $ --- работа выхода электрона из катода. Реально энергетический спектр вылетевших из фотокатода электронов непрерывен --- он простирается от нуля до $ E_max $. 
	
	Для измерения энергии вылетевших фотоэлектронов вблизи фотокатода
	обычно располагается второй электрод
	(анод), на который подается задерживающий ($ V < 0 $) или ускоряющий ($ V >
	0 $) потенциал. При достаточно больших
	ускоряющих напряжениях фототок достигает насыщения (рис. \ref{ris I(V)}): все испущенные электроны попадают на анод.
	
	При задерживающих потенциалах на анод попадают лишь электроны,
	обладающие достаточно большой кинетической энергией, в то время
	как медленно движущиеся электроны заворачиваются полем и возвращаются на катод. При некотором значении $ V = -V0 $ (потенциал запирания) даже наиболее быстрые фотоэлектроны не могут достичь
	анода.
	Максимальная кинетическая энергия $ E_{max} $ электронов связана с
	запирающим потенциалом $ V0 $ очевидным соотношением $ E_{max} = eV_0 $. Тогда \eqref{energy balance} примет вид, называемый уравнением Эйнштейна:
	
	\begin{equation}\label{Einsteain}
	eV_0	\hbar\omega - W 
	\end{equation}
	
	Чтобы определить величину запирающего
	напряжения, нам надо правильно экстраполировать получаемую токовую зависимость к нулю, т. е. определить, какова функциональная
	зависимость $ I(V ) $. Расчет для простейшей геометрии --- плоский катод, освещаемый светом, и параллельный ему анод --- приводит к зависимости
	
	\begin{equation}\label{sqrt I = V}
	\sqrt{I} \propto V_0 - V
	\end{equation}
	
	т. е. корень квадратный из фототока линейно
	зависит от запирающего напряжения. Эта зависимость хорошо описывает экспериментальные данные.
	
	В работе изучается зависимость фототока из фотоэлемента от величины задерживающего потенциала $ V $ для различных частот света $ \omega $, лежащих в видимой области спектра. С целью экспериментальной
	проверки уравнения Эйнштейна определяются потенциалы запирания
	$ V_0 $ при разных частотах света и строится зависимость $ V_0(\omega) $, которая, как это следует из \eqref{Einsteain}, должна иметь вид
	
	\begin{equation}\label{V(w)}
	V_0 (\omega) = \dfrac{\hbar\omega - W}{e}
	\end{equation}
	
		\begin{wrapfigure}{r}{0.3\linewidth}
		\includegraphics[width=\linewidth]{V(w)}
		\caption{Зависимость запирающего потенциала
			от частоты света}
		\label{ris V(w)}
	\end{wrapfigure}
	
	Потенциал запирания $ V_0 $ для любого катода линейно зависит от
	частоты света $ \omega $. По наклону прямой на графике $ V_0(\omega) $ (рис. \ref{ris V(w)}) можно определить постоянную Планка:
	
	\begin{equation}\label{dV/dw}
	\dfrac{dV_0}{d\omega} = \dfrac{\hbar}{e}
	\end{equation}
	
	Как показывает формула \eqref{dV/dw}, угол наклона прямой $ V_0(\omega) $ не зависит от рода вещества, из которого изготовлен фотокатод. От рода вещества, однако, зависит величина фототока, работа выхода $ W $ и форма кривой $ I(V) $ (рис. \ref{ris I(V)}). Все это определяет выбор пригодных для
	опыта катодов.


%	\section{Экспериментальная установка}
	
	
	\section{Выполнение работы}
	
	
	
	
	\section{Вывод }
	
\end{document}
