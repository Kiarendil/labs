\documentclass[12pt]{kiarticle}
\graphicspath{{pictures/}}
\DeclareGraphicsExtensions{.pdf,.png,.jpg,.eps}
%%%
\pagestyle{fancy}
\fancyhf{}
%\renewcommand{\headrulewidth}{ 0.1mm }
\renewcommand{\footrulewidth}{ .0em }
\fancyfoot[C]{\texttt{\textemdash~\thepage~\textemdash}}
\fancyhead[L]{Лабораторная работа № 4.2 \hfil}
\fancyhead[R]{\hfil Иванов Кирилл, 625 группа }
\usepackage{multirow} % Слияние строк в таблице
\newcommand
{\un}[1]
{\ensuremath{\text{#1}}}
\newcommand{\eds}{\ensuremath{ \mathscr{E}}}
\newcommand{\be}{\ensuremath{\beta}}
\usepackage{tikz}
%%% Работа с таблицами
\usepackage{array,tabularx,tabulary,booktabs} % Дополнительная работа с таблицами
\usepackage{longtable}  % Длинные таблицы
\usepackage{multirow} % Слияние строк в таблице

\begin{document}
	
	\begin{titlepage}
		\begin{center}
			\large 	Московский физико-технический институт \\
			(государственный университет) \\
			Факультет общей и прикладной физики \\
			\vspace{0.2cm}
			
			\vspace{4.5cm}
			Лабораторная работа № 4.2 \\ \vspace{0.2cm}
			\large (Общая физика: квантовая физика) \\ \vspace{0.2cm}
			\LARGE \textbf{ Исследование энергетического спектра \be-частиц
				и определение их максимальной энергии при помощи
				магнитного спектрометра }
		\end{center}
		\vspace{2.3cm} \large
		
		\begin{center}
			Работу выполнил: \\
			Иванов Кирилл,
			625 группа
			\vspace{10mm}		
			
		\end{center}
		
		\begin{center} \vspace{60mm}
			г. Долгопрудный \\
			2018 год
		\end{center}
	\end{titlepage}


\paragraph*{Цель работы:} С помощью магнитного спектрометра исследовать энергетический спектр $\beta$ - частиц при распаде ядер $^{137}$Cs  и определить их максимальную энергию.

\section{Теоретическое введение} 

Бета-распадом называется самопроизвольное превращение ядер, при котором их массовое число не изменяется, а заряд увеличивается или уменьшается на единицу. Бета-активные ядра встречаются во всей области значений массового числа A, начиная от единицы (свободный нейтрон) и кончая самыми тяжелыми ядрами. Период полураспада $\beta$ - активных ядер изменяется от ничтожных долей секунды до $10^{18}$ лет. Выделяющаяся при единичном акте $\beta$ - распада энергия варьируется от 18 кэВ до 13,4 МэВ.

В данной работе мы будем иметь дело с электронным распадом

\begin{equation}\label{}
^A_ZX \rightarrow ^{\; \; \; \; \: A}_{Z+1}X + e^- + \widetilde{\nu}
\end{equation}

при котором кроме электрона испускается антинейтрино. Освобождающаяся при $\beta$-распаде энергия делится между электроном, антинейтрино и дочерним ядром, однако доля энергии, передаваемой ядру, исчезающе мала по сравнению с энергией, уносимой электроном и антинейтрино. Практически можно считать, что эти две частицы делят между собой всю освобождающуюся энергию. Поэтому электроны могут иметь любое значение энергии  от нулевой до некоторой максимальной, которая равна энергии, освобождающейся при $\beta$-распаде, являющейся важной физической величиной.

Вероятность $ dw $ того, что при распаде электрон вылетит с импульсом в интервале $d^3p$, а антинейтрино с импульсом в интервале $d^3k$, пропорциональна произведению этих дифференциалов. Но мы должны еще учесть закон сохранения энергии, согласно которому импульсы $ p $ и $ k $ электрона и антинейтрино связаны соотношением

\begin{equation}
E_e - E - ck = 0,
\end{equation}

где $E_e$ - максимальная энергия электрона, кинетическая энергия электрона $ E $ связана с его импульсом обычным релятивистским соотношением

\begin{equation}
E = c\sqrt{p^2 + m^2c^2} - mc^2,
\end{equation}

а через $ ck $ обозначена энергия антинейтрино с импульсом $ k $. Условие можно учесть введением в выражение для $ dw $ $\delta$ - функции
\begin{equation}
\delta (E_e - E - ck).
\end{equation}

Таким образом, вероятность $ dw $ может быть записана в виде

\begin{equation}\label{dw}
dw = D \delta (E_e - E - ck)d^3 p d^3 k = D \delta (E_e - E - ck)p^2dpk^2dkd{\Omega}_ed{\Omega}_{\widetilde{\nu}},
\end{equation}

где $ D $ --- некоторый коэффициент пропорциональности, $d\Omega_e$ , $d\Omega_{\widetilde{\nu}}$ --- элементы телесных углов направлений вылета электрона и нейтрино. Вероятность $ dw $ непосредственно связана с $\beta$-спектром, поскольку для большого числа $N_0$ распадов число $dN$ распадов с вылетом электрона и антинейтрино с импульсом соответственно от $ p $ до $ p + dp $ и от
$ k $ до $ k + d $k определяется соотношением

\begin{equation}\label{dn = n_0 dw}
dN = N_0 dw  
\end{equation}

Коэффициент $ D $ в формуле \eqref{dw} можно считать для рассматриваемых нами так называемых разрешенных фермиевских типов распадов с хорошей точностью константой (разрешенными называются такие переходы, при которых не изменяются ни момент, ни четность состояния ядра). В этом случае величину $ dw $ из \eqref{dn = n_0 dw} можно проинтегрировать по всем углам и по абсолютному значению импульса нейтрино.

После умножения на полное число распадов $ N $ проинтегрированное выражение приобретает смысл числа электронов $ dN $, вылетающих из ядра с импульсом, абсолютная величина которого лежит между $ p $ и$  p + dp $:

\begin{equation}\label{dN}
dN = \dfrac{16\pi^2 N_0}{c^2}Dp^2(E_e - E)^2dp.
\end{equation}

Чтобы получить распределение электронов по энергиям, надо в \eqref{dN} перейти от $ dp $ к $ dE $:

\begin{equation}
dE = \dfrac{c^2p}{E + mc^2}dp,
\end{equation}

после чего выражающая форму $\beta$ --- спектра величина $ N(E) = dN/dE $
приобретает вид

\begin{equation}\label{dN/dE big}
\dfrac{dN}{dE} = N_0Bcp(E + mc^2)(E_e - E)^2 = N_0B\sqrt{E(E + 2mc^2)}(E_e - E)^2(E + mc^2)
\end{equation}

где $B = (16\pi^2/c^4)D$. В нерелятивистском приближении, которое и имеет место с нашем случае, выражение \eqref{dN/dE big} упрощается, и мы имеем

\begin{equation}\label{dN/dE}
\dfrac{dN}{dE} \approx \sqrt{E}(E_e - E)^2.
\end{equation}

\begin{wrapfigure}[12]{l}{0.35\linewidth}
	\includegraphics[width=\linewidth]{spektr}
	\caption{Форма спектра \be-частиц
		при разрешенных переходах}
	\label{ris spetr}
\end{wrapfigure}


Выражение \eqref{dN/dE} приводит к спектру, имеющему вид широкого колокола (рис \ref{ris spetr}). Кривая плавно отходит от нуля и столь же плавно, по параболе, касается оси абсцисс в области максимальной энергии электронов $E_e$. 

Дочерние ядра, возникающие в результате $\beta$-распада, нередко оказываются возбужденными. Возбужденные ядра отдают свою энергию либо излучая $\gamma$-квант (энергия которого равна разности энергий начального и конечного уровней), либо передавая избыток энергии одному из электронов с внутренних оболочек атома. Излучаемые в таком процессе электроны имеют строго определенную энергию и называются конверсионными.

Конверсия чаще всего происходит на оболочках $ K $ или $ L $. На спектре, представленном на рис. \ref{ris spetr}, видна монохроматическая линия, вызванная электронами конверсии. Ширина этой линии в нашем случае является чисто аппаратурной, по ней можно оценить разрешающую силу спектрометра.

\section{Выполнение работы}

Откачаем воздух из полости спектрометра, включим вакуумметр. Включим ПЭВМ, формирователь импульсов, питание магнитной линзы и уменьшим ток через неё до нуля. 

Проведём измерение $\beta$-спектра, изменяя ток в магнитной линзе, при каждом значении тока будем измерять число попаданий частиц в детектор за 100 секунд, результаты занесём в таблицу.

Измерим фон:
	
	
	
	\section{Выполнение работы}
	
	
	\section{Вывод}
	
	
	
\end{document}
